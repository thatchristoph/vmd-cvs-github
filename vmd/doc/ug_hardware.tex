%%%%%%%%%%%%%%%%%%%%%%%%%%%%%%%%%%%%%%%%%%%%%%%%%%%%%%%%%%%%%%%%%%%%%%%%%%%%
% RCS INFORMATION:
%
%       $RCSfile: ug_hardware.tex,v $
%       $Author: johns $        $Locker:  $                $State: Exp $
%       $Revision: 1.1 $      $Date: 2009/07/31 06:06:51 $
%
%%%%%%%%%%%%%%%%%%%%%%%%%%%%%%%%%%%%%%%%%%%%%%%%%%%%%%%%%%%%%%%%%%%%%%%%%%%%
% DESCRIPTION:
%   Basic hardware requirements, features, etc..               
%
%%%%%%%%%%%%%%%%%%%%%%%%%%%%%%%%%%%%%%%%%%%%%%%%%%%%%%%%%%%%%%%%%%%%%%%%%%%%

\chapter{Hardware and Software Requirements}
\label{chapter:requirements:ug}
\index{hardware requirements}

\section{Basic Hardware and Software Requirements}
The basic hardware requirements for running VMD vary depending on 
how it was compiled and how it will be used.  VMD has two primary
modes of operation, the typical full-featured graphics-enabled mode, 
and a purely text-based mode of operation suited for remote analysis
on supercomputers, embedded use in other packages, and similar 
batch-oriented analytical uses.  

The full-featured graphics-enabled mode of VMD is the most demanding,
and requires an OpenGL-capable graphics accelerator with up-to-date 
drivers.  Some graphics chipsets or GPUs come with drivers that are
below-spec and will not be able to run VMD with full graphics capability.
These will either automatically, or as the result of 
user-defined environment variables (e.g. VMDSIMPLEGRAPHICS), 
use a reduced functionality graphics mode within \VMD.
Since the choice of the GPU chipset or card has the biggest
impact on the visualization capabilities and performance of VMD,
this is the hardware component that is worth spending money on 
if one's intended use of \VMD\ is primarily focused on visualization related
tasks.  As an added bonus, recent GPUs are now also capable of accelerating
some of the computationally demanding tasks within VMD, as will be discussed
in more detail below.

Following the choice of graphics accelerator, the amount of available
system memory tends to have the next most significant impact on the
performance and capability of VMD.  The more memory a machine has,
the more frames can be loaded at once from large molecular
dynamics trajectory files.  For batch-mode analysis tasks 
that consist primarily of scripting, system memory is frequently the
resource that limits feasability of many analysis tasks.

\section{Multi-core CPUs and GPU Acceleration}
VMD makes full use of multi-core processors and multiple GPUs for 
acceleration of the most computationally demanding visualization 
and analysis tasks.  Multi-core CPUs accelerate features including
interactive MD, bond determination, ``within'' atom selections 
and derivatives, radial distribution functions, and high quality
renderings using the ``Tachyon'' and ``TachyonInternal'' ray tracing 
features.  \VMD\ also supports GPU acceleration using CUDA, and takes 
advantage of both multi-core CPUs and GPUs for acceleration of 
electrostatics (i.e. ``volmap coulomb'', and ``volmap coulombmsm''), 
implicit ligand sampling (i.e. ``volmap ils''), and 
computation and rendering of molecular orbitals.


